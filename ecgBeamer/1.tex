\section[Introducción]{Introducción}
\begin{frame}[plain]
	\frametitle{Descripción del problema}
	
	
	\begin{defn}
	Sea un vector \textit{v} de números de tamaño \textit{n}, todos distintos, de forma que existe un índice \textit{p} (que no es ni el primero ni el último) tal que a la \textbf{izquierda de 	\textit{p} } los números están ordenados de forma \textbf{creciente} y a la \textbf{derecha} de \textit{p} están ordenados de forma \textbf{decreciente}; es decir \\
	
		\begin{equation}
			 \forall i,j \leq p, i < j \Rightarrow v[i] < v[j]  \hspace*{0.18in} \textbf{y} \hspace*{0.18in}  \forall i,j \geq p, i < j \Rightarrow v[i] > v[j] 
		\end{equation}
		\vspace*{0.05in}
		
	\end{defn}
\end{frame}
	
	
%	\begin{block}{Algoritmos que vamos a estudiar}
%			\begin{itemize}[<+-| alert@+>]
%				\item Burbuja
%				\item Inserción
%				\item Selección
%				\item Mergesort
%				\item Quicksort
%				\item Heapsort
%				\item Floyd
%				\item Hanoi
%			\end{itemize}	
%		\end{block}


%\input{Images/Explicacion}
	
	
%	%Contenido
%		\begin{block}{Título de bloque}
%			\begin{itemize}[<+-| alert@+>]
%				\item Gerardo es tonto \pause
%				\item Gerardo es más tonto aún \pause
%				\item Gerardo es gilipollas
%			\end{itemize}	
%		\end{block}
%		\pause
%
%		\begin{alertblock}{Importante}
%			Quien no tenga hoy hecho  el constructor por defecto de la clase fecha, ¡No aprueba conmigo!
%		\end{alertblock}
%		
%		\pause
%		
%		\begin{exampleblock}{Ejemplo}
%			De alguna manera...
%		\end{exampleblock}
%		
%		\pause


	
