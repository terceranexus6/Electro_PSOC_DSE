\section[Algoritmo Floyd]{Algoritmo Floyd}
\begin{frame}[plain]
%	\frametitle{Floyd}
%	
%	
%			\begin{defn}
%			
%			Este algoritmo  intenta encontrar el camino más corto entre todos los pares de nodos o vértices de un grafo. Esto es semejante a construir una tabla (matriz) con todas las distancias mínimas entre pares de ciudades de un mapa, indicando además la ruta a seguir para ir de la primera ciudad a la segunda. Este es uno de los problemas más interesantes que se pueden resolver con algoritmos de grafos.
%			\end{defn}
%		
%		\begin{equation}\label{eq:cubico}
%				T(n)=a\cdot n^{3}+b\cdot n^{2} + c \cdot n + d 
%			\end{equation}
		
\end{frame}
		
\begin{frame}[plain]
%	\frametitle{Floyd - Análisis gráfico}
%	
%		\begin{figure}[htb]
%		\begin{center}
%		\begin{picture}(130,0)
%		\put(-50,-110){\includegraphics[width=8.45cm,height=6.5cm]{images/floyd-empirico-ivan}}
%		\end{picture}
%		\end{center}
%		\end{figure}
%	
	
\end{frame}	