\section[Algoritmo Heapsort]{Algoritmo Heapsort}
\begin{frame}[plain]
%	\frametitle{Heapsort}
%	
%	
%			\begin{defn}
%			Este algoritmo consiste en almacenar todos los elementos del vector a ordenar en un montículo (heap), y luego extraer el nodo que queda como nodo raíz del montículo (cima) en sucesivas iteraciones obteniendo el conjunto ordenado.   \\ 
%				 El algoritmo, después de cada extracción, recoloca en el nodo raíz o cima, la última hoja por la derecha del último nivel. Lo cual destruye la propiedad heap del árbol. Pero, a continuación realiza un proceso de descenso del número insertado de forma que se elige a cada movimiento el mayor de sus dos hijos, con el que se intercambia. Este intercambio, realizado sucesivamente hunde el nodo en el árbol restaurando la propiedad montículo del arbol y dejándo paso a la siguiente extracción del nodo raíz.
%			
%		\end{defn}
%		
%		\begin{equation}\label{eq:heapsort}
%				T(n) = a \cdot n \cdot log(n) + b
%		\end{equation}
		
\end{frame}
		
\begin{frame}[plain]
%	\frametitle{Heapsort - Análisis gráfico}
%	
%		\begin{figure}[htb]
%		\begin{center}
%		\begin{picture}(130,0)
%		\put(-50,-110){\includegraphics[width=8.45cm,height=6.5cm]{images/heapsort-empirico-ivan}}
%		\end{picture}
%		\end{center}
%		\end{figure}
%	
%	
\end{frame}	