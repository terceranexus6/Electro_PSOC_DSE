\section[Algoritmo Mergesort]{Algoritmo Mergesort}
\begin{frame}[plain]
%	\frametitle{Mergesort}
%	
%	\begin{defn}
%			Conceptualmente, el ordenamiento por mezcla funciona de la siguiente manera. Si la longitud de la lista es 0 ó 1, entonces ya está ordenada. En otro caso:
%
%			\begin{itemize}
%				\item Dividir la lista desordenada en dos sublistas de aproximadamente la mitad del tamaño.
%				\item Ordenar cada sublista recursivamente aplicando el ordenamiento por mezcla.
%				\item Mezclar las dos sublistas en una sola lista ordenada.
%
%			\end{itemize}
%
%			
%		\end{defn}
%		
%		
%		\begin{equation}\label{eq:mergesort}
%				T(n) = a \cdot n \cdot log(n) + b
%			\end{equation}
		
\end{frame}
		
\begin{frame}[plain]
%	\frametitle{Mergesort - Análisis gráfico}
%	
%		\begin{figure}[htb]
%		\begin{center}
%		\begin{picture}(130,0)
%		\put(-50,-110){\includegraphics[width=8.45cm,height=6.5cm]{images/mergesort-empirico-ivan}}
%		\end{picture}
%		\end{center}
%		\end{figure}
%	
	
\end{frame}	