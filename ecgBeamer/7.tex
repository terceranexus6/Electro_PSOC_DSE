\section[Algoritmo Quicksort]{Algoritmo Quicksort}
\begin{frame}[plain]
%	\frametitle{Quicksort}
%	
%		
%		
%			\begin{defn}
%			El algoritmo quicksort funciona de la siguiente forma:
%
%			\begin{itemize}
%				\item Elegir un elemento de la lista de elementos a ordenar, al que llamaremos pivote.
%				\item Resituar los demás elementos de la lista a cada lado del pivote, de manera que a un lado queden todos los menores que él, y al otro los mayores. Los elementos iguales al pivote pueden ser colocados tanto a su derecha como a su izquierda, dependiendo de la implementación deseada. En este momento, el pivote ocupa exactamente el lugar que le corresponderá en la lista ordenada.
%				\item La lista queda separada en dos sublistas, una formada por los elementos a la izquierda del pivote, y otra por los elementos a su derecha.
%				\item Repetir este proceso de forma recursiva para cada sublista mientras éstas contengan más de un elemento. Una vez terminado este proceso todos los elementos estarán ordenados.
%
%			\end{itemize}
%
%			
%		\end{defn}
%		
%		
%		
%		\begin{equation}\label{eq:quicksort}
%				T(n) = a \cdot n \cdot log(n) + b
%			\end{equation}
%		
\end{frame}
		
\begin{frame}[plain]
%	\frametitle{Quicksort - Análisis gráfico}
%	
%		\begin{figure}[htb]
%		\begin{center}
%		\begin{picture}(130,0)
%		\put(-50,-110){\includegraphics[width=8.45cm,height=6.5cm]{images/quicksort-empirico-ivan}}
%		\end{picture}
%		\end{center}
%		\end{figure}
%	
	
\end{frame}	